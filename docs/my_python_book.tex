% Options for packages loaded elsewhere
\PassOptionsToPackage{unicode}{hyperref}
\PassOptionsToPackage{hyphens}{url}
%
\documentclass[
]{book}
\usepackage{amsmath,amssymb}
\usepackage{lmodern}
\usepackage{iftex}
\ifPDFTeX
  \usepackage[T1]{fontenc}
  \usepackage[utf8]{inputenc}
  \usepackage{textcomp} % provide euro and other symbols
\else % if luatex or xetex
  \usepackage{unicode-math}
  \defaultfontfeatures{Scale=MatchLowercase}
  \defaultfontfeatures[\rmfamily]{Ligatures=TeX,Scale=1}
\fi
% Use upquote if available, for straight quotes in verbatim environments
\IfFileExists{upquote.sty}{\usepackage{upquote}}{}
\IfFileExists{microtype.sty}{% use microtype if available
  \usepackage[]{microtype}
  \UseMicrotypeSet[protrusion]{basicmath} % disable protrusion for tt fonts
}{}
\makeatletter
\@ifundefined{KOMAClassName}{% if non-KOMA class
  \IfFileExists{parskip.sty}{%
    \usepackage{parskip}
  }{% else
    \setlength{\parindent}{0pt}
    \setlength{\parskip}{6pt plus 2pt minus 1pt}}
}{% if KOMA class
  \KOMAoptions{parskip=half}}
\makeatother
\usepackage{xcolor}
\IfFileExists{xurl.sty}{\usepackage{xurl}}{} % add URL line breaks if available
\IfFileExists{bookmark.sty}{\usepackage{bookmark}}{\usepackage{hyperref}}
\hypersetup{
  pdftitle={Python Scracth Book},
  pdfauthor={DEA},
  hidelinks,
  pdfcreator={LaTeX via pandoc}}
\urlstyle{same} % disable monospaced font for URLs
\usepackage{color}
\usepackage{fancyvrb}
\newcommand{\VerbBar}{|}
\newcommand{\VERB}{\Verb[commandchars=\\\{\}]}
\DefineVerbatimEnvironment{Highlighting}{Verbatim}{commandchars=\\\{\}}
% Add ',fontsize=\small' for more characters per line
\usepackage{framed}
\definecolor{shadecolor}{RGB}{248,248,248}
\newenvironment{Shaded}{\begin{snugshade}}{\end{snugshade}}
\newcommand{\AlertTok}[1]{\textcolor[rgb]{0.94,0.16,0.16}{#1}}
\newcommand{\AnnotationTok}[1]{\textcolor[rgb]{0.56,0.35,0.01}{\textbf{\textit{#1}}}}
\newcommand{\AttributeTok}[1]{\textcolor[rgb]{0.77,0.63,0.00}{#1}}
\newcommand{\BaseNTok}[1]{\textcolor[rgb]{0.00,0.00,0.81}{#1}}
\newcommand{\BuiltInTok}[1]{#1}
\newcommand{\CharTok}[1]{\textcolor[rgb]{0.31,0.60,0.02}{#1}}
\newcommand{\CommentTok}[1]{\textcolor[rgb]{0.56,0.35,0.01}{\textit{#1}}}
\newcommand{\CommentVarTok}[1]{\textcolor[rgb]{0.56,0.35,0.01}{\textbf{\textit{#1}}}}
\newcommand{\ConstantTok}[1]{\textcolor[rgb]{0.00,0.00,0.00}{#1}}
\newcommand{\ControlFlowTok}[1]{\textcolor[rgb]{0.13,0.29,0.53}{\textbf{#1}}}
\newcommand{\DataTypeTok}[1]{\textcolor[rgb]{0.13,0.29,0.53}{#1}}
\newcommand{\DecValTok}[1]{\textcolor[rgb]{0.00,0.00,0.81}{#1}}
\newcommand{\DocumentationTok}[1]{\textcolor[rgb]{0.56,0.35,0.01}{\textbf{\textit{#1}}}}
\newcommand{\ErrorTok}[1]{\textcolor[rgb]{0.64,0.00,0.00}{\textbf{#1}}}
\newcommand{\ExtensionTok}[1]{#1}
\newcommand{\FloatTok}[1]{\textcolor[rgb]{0.00,0.00,0.81}{#1}}
\newcommand{\FunctionTok}[1]{\textcolor[rgb]{0.00,0.00,0.00}{#1}}
\newcommand{\ImportTok}[1]{#1}
\newcommand{\InformationTok}[1]{\textcolor[rgb]{0.56,0.35,0.01}{\textbf{\textit{#1}}}}
\newcommand{\KeywordTok}[1]{\textcolor[rgb]{0.13,0.29,0.53}{\textbf{#1}}}
\newcommand{\NormalTok}[1]{#1}
\newcommand{\OperatorTok}[1]{\textcolor[rgb]{0.81,0.36,0.00}{\textbf{#1}}}
\newcommand{\OtherTok}[1]{\textcolor[rgb]{0.56,0.35,0.01}{#1}}
\newcommand{\PreprocessorTok}[1]{\textcolor[rgb]{0.56,0.35,0.01}{\textit{#1}}}
\newcommand{\RegionMarkerTok}[1]{#1}
\newcommand{\SpecialCharTok}[1]{\textcolor[rgb]{0.00,0.00,0.00}{#1}}
\newcommand{\SpecialStringTok}[1]{\textcolor[rgb]{0.31,0.60,0.02}{#1}}
\newcommand{\StringTok}[1]{\textcolor[rgb]{0.31,0.60,0.02}{#1}}
\newcommand{\VariableTok}[1]{\textcolor[rgb]{0.00,0.00,0.00}{#1}}
\newcommand{\VerbatimStringTok}[1]{\textcolor[rgb]{0.31,0.60,0.02}{#1}}
\newcommand{\WarningTok}[1]{\textcolor[rgb]{0.56,0.35,0.01}{\textbf{\textit{#1}}}}
\usepackage{longtable,booktabs,array}
\usepackage{calc} % for calculating minipage widths
% Correct order of tables after \paragraph or \subparagraph
\usepackage{etoolbox}
\makeatletter
\patchcmd\longtable{\par}{\if@noskipsec\mbox{}\fi\par}{}{}
\makeatother
% Allow footnotes in longtable head/foot
\IfFileExists{footnotehyper.sty}{\usepackage{footnotehyper}}{\usepackage{footnote}}
\makesavenoteenv{longtable}
\usepackage{graphicx}
\makeatletter
\def\maxwidth{\ifdim\Gin@nat@width>\linewidth\linewidth\else\Gin@nat@width\fi}
\def\maxheight{\ifdim\Gin@nat@height>\textheight\textheight\else\Gin@nat@height\fi}
\makeatother
% Scale images if necessary, so that they will not overflow the page
% margins by default, and it is still possible to overwrite the defaults
% using explicit options in \includegraphics[width, height, ...]{}
\setkeys{Gin}{width=\maxwidth,height=\maxheight,keepaspectratio}
% Set default figure placement to htbp
\makeatletter
\def\fps@figure{htbp}
\makeatother
\setlength{\emergencystretch}{3em} % prevent overfull lines
\providecommand{\tightlist}{%
  \setlength{\itemsep}{0pt}\setlength{\parskip}{0pt}}
\setcounter{secnumdepth}{5}
\usepackage{booktabs}
\usepackage{amsthm}
\makeatletter
\def\thm@space@setup{%
  \thm@preskip=8pt plus 2pt minus 4pt
  \thm@postskip=\thm@preskip
}
\makeatother
\ifLuaTeX
  \usepackage{selnolig}  % disable illegal ligatures
\fi
\usepackage[]{natbib}
\bibliographystyle{apalike}

\title{Python Scracth Book}
\author{DEA}
\date{2022-07-04}

\begin{document}
\maketitle

{
\setcounter{tocdepth}{1}
\tableofcontents
}
\hypertarget{acknowledgement}{%
\chapter{Acknowledgement}\label{acknowledgement}}

\hypertarget{intro}{%
\chapter{Introduction}\label{intro}}

\textbf{Jupyter Lab}

to install on mac: \texttt{pip3\ install\ jupyterlab} on terminal

to upgrade pip: \texttt{pip3\ install\ -\/-upgrade\ pip} on terminal

\textbf{check python version}

\begin{Shaded}
\begin{Highlighting}[]
\OperatorTok{!}\NormalTok{python }\OperatorTok{{-}}\NormalTok{V}
\end{Highlighting}
\end{Shaded}

\textbf{Python}

\texttt{Python} is what is called an interpreted language. Compiled languages examine your entire program at compile time, and are able to warn you about a whole class of errors prior to execution. In contrast, Python interprets your script line by line as it executes it. Python will stop executing the entire program when it encounters an error (unless the error is expected and handled by the programmer, a more advanced subject that we'll cover later on in this course).

\begin{Shaded}
\begin{Highlighting}[]
\CommentTok{\# Check the Python Version}

\ImportTok{import}\NormalTok{ sys}
\BuiltInTok{print}\NormalTok{(sys.version)}
\end{Highlighting}
\end{Shaded}

\begin{verbatim}
## 3.10.3 (v3.10.3:a342a49189, Mar 16 2022, 09:34:18) [Clang 13.0.0 (clang-1300.0.29.30)]
\end{verbatim}

{[}Tip:{]} \texttt{sys} is a built-in module that contains many system-specific parameters and functions, including the Python version in use. Before using it, we must explicitly import it.

\hypertarget{types-of-objects-in-python}{%
\section{Types of objects in Python}\label{types-of-objects-in-python}}

Python is an object-oriented language. There are many different types of objects in Python. Let's start with the most common object types: strings, integers and floats. Anytime you write words (text) in Python, you're using character strings (strings for short). The most common numbers, on the other hand, are integers (e.g.~-1, 0, 100) and floats, which represent real numbers (e.g.~3.14, -42.0).

\textbf{Object Types}

\begin{itemize}
\tightlist
\item
  integer = 10
\item
  float = 10.1
\item
  string = ``Hello''
\item
  boolean = True
\end{itemize}

\textbf{float to integer}

\begin{Shaded}
\begin{Highlighting}[]
\NormalTok{a }\OperatorTok{=} \FloatTok{10.123}
\BuiltInTok{type}\NormalTok{(a)}
\end{Highlighting}
\end{Shaded}

\begin{verbatim}
## <class 'float'>
\end{verbatim}

\begin{Shaded}
\begin{Highlighting}[]
\NormalTok{b }\OperatorTok{=} \BuiltInTok{int}\NormalTok{(a)}
\NormalTok{b}
\end{Highlighting}
\end{Shaded}

\begin{verbatim}
## 10
\end{verbatim}

\begin{Shaded}
\begin{Highlighting}[]
\BuiltInTok{type}\NormalTok{(b)}
\end{Highlighting}
\end{Shaded}

\begin{verbatim}
## <class 'int'>
\end{verbatim}

\textbf{numeric to string}

\begin{Shaded}
\begin{Highlighting}[]
\NormalTok{a\_string }\OperatorTok{=} \BuiltInTok{str}\NormalTok{(a)}
\NormalTok{a\_string}
\end{Highlighting}
\end{Shaded}

\begin{verbatim}
## '10.123'
\end{verbatim}

\textbf{string to numeric}

\begin{Shaded}
\begin{Highlighting}[]
\BuiltInTok{float}\NormalTok{(}\StringTok{"1.1"}\NormalTok{)}
\end{Highlighting}
\end{Shaded}

\begin{verbatim}
## 1.1
\end{verbatim}

it does not transform directly to integer here

\begin{Shaded}
\begin{Highlighting}[]
\BuiltInTok{int}\NormalTok{(}\StringTok{"1.123"}\NormalTok{)}
\end{Highlighting}
\end{Shaded}

but it works when transforming to float then integer.

\begin{Shaded}
\begin{Highlighting}[]
\BuiltInTok{int}\NormalTok{(}\BuiltInTok{float}\NormalTok{(}\StringTok{"10.123"}\NormalTok{))}
\end{Highlighting}
\end{Shaded}

\textbf{boolean}

\begin{Shaded}
\begin{Highlighting}[]
\NormalTok{bl }\OperatorTok{=} \VariableTok{True}
\NormalTok{bl}
\end{Highlighting}
\end{Shaded}

\begin{verbatim}
## True
\end{verbatim}

\begin{Shaded}
\begin{Highlighting}[]
\BuiltInTok{type}\NormalTok{(bl)}
\end{Highlighting}
\end{Shaded}

\begin{verbatim}
## <class 'bool'>
\end{verbatim}

\textbf{boolean to numeric}

\texttt{True} becomes 1

\begin{Shaded}
\begin{Highlighting}[]
\BuiltInTok{int}\NormalTok{(bl)}
\end{Highlighting}
\end{Shaded}

\begin{verbatim}
## 1
\end{verbatim}

\textbf{numeric to boolean}

0 becomes \texttt{False}
all other numbers are \texttt{True}

\begin{Shaded}
\begin{Highlighting}[]
\BuiltInTok{bool}\NormalTok{(}\OperatorTok{{-}}\DecValTok{100}\NormalTok{)}
\end{Highlighting}
\end{Shaded}

\begin{verbatim}
## True
\end{verbatim}

\begin{Shaded}
\begin{Highlighting}[]
\BuiltInTok{bool}\NormalTok{(}\DecValTok{0}\NormalTok{)}
\end{Highlighting}
\end{Shaded}

\begin{verbatim}
## False
\end{verbatim}

\hypertarget{expressions-mathematical-operations}{%
\section{Expressions: Mathematical Operations}\label{expressions-mathematical-operations}}

\begin{Shaded}
\begin{Highlighting}[]
\DecValTok{5} \OperatorTok{+} \DecValTok{5} \OperatorTok{*} \DecValTok{10}  \OperatorTok{{-}} \DecValTok{2} \OperatorTok{/} \DecValTok{5}
\end{Highlighting}
\end{Shaded}

\begin{verbatim}
## 54.6
\end{verbatim}

\textbf{integer division}

\begin{Shaded}
\begin{Highlighting}[]
\DecValTok{11} \OperatorTok{//} \DecValTok{2}   
\end{Highlighting}
\end{Shaded}

\begin{verbatim}
## 5
\end{verbatim}

\textbf{modulo: remainder}

\begin{Shaded}
\begin{Highlighting}[]
\DecValTok{10} \OperatorTok{\%} \DecValTok{3}   
\end{Highlighting}
\end{Shaded}

\begin{verbatim}
## 1
\end{verbatim}

\hypertarget{string-operations}{%
\section{String Operations}\label{string-operations}}

-string object are in single quote or double quote

\hypertarget{indices}{%
\subsection{Indices}\label{indices}}

\textbf{Positive Index}
In python \textbf{indices} start with 0
In R \textbf{indices} start with 1

\textbf{Negative Index}
\texttt{-1} corresponds to the last element

\begin{Shaded}
\begin{Highlighting}[]
\NormalTok{myname }\OperatorTok{=} \StringTok{"Davut Emrah Ayan"}

\BuiltInTok{print}\NormalTok{(}\StringTok{"Object myname:"}\NormalTok{, myname)}
\CommentTok{\# examples}
\end{Highlighting}
\end{Shaded}

\begin{verbatim}
## Object myname: Davut Emrah Ayan
\end{verbatim}

\begin{Shaded}
\begin{Highlighting}[]
\BuiltInTok{print}\NormalTok{(myname[}\DecValTok{0}\NormalTok{], }\StringTok{\textquotesingle{}is the first element of myname object\textquotesingle{}}\NormalTok{)}
\end{Highlighting}
\end{Shaded}

\begin{verbatim}
## D is the first element of myname object
\end{verbatim}

\begin{Shaded}
\begin{Highlighting}[]
\BuiltInTok{print}\NormalTok{(myname[}\DecValTok{6}\NormalTok{], }\StringTok{\textquotesingle{}is the 6th element of myname object\textquotesingle{}}\NormalTok{)}
\end{Highlighting}
\end{Shaded}

\begin{verbatim}
## E is the 6th element of myname object
\end{verbatim}

\begin{Shaded}
\begin{Highlighting}[]
\BuiltInTok{print}\NormalTok{(myname[}\OperatorTok{{-}}\DecValTok{1}\NormalTok{], }\StringTok{\textquotesingle{}is the last element of myname object\textquotesingle{}}\NormalTok{)}
\end{Highlighting}
\end{Shaded}

\begin{verbatim}
## n is the last element of myname object
\end{verbatim}

\hypertarget{slicing}{%
\subsection{Slicing}\label{slicing}}

\textbf{Full version}

object{[}from : to : increment{]}

\begin{Shaded}
\begin{Highlighting}[]
\BuiltInTok{print}\NormalTok{(}\StringTok{"Object:"}\NormalTok{, myname)}
\end{Highlighting}
\end{Shaded}

\begin{verbatim}
## Object: Davut Emrah Ayan
\end{verbatim}

\begin{Shaded}
\begin{Highlighting}[]
\BuiltInTok{print}\NormalTok{(}\StringTok{"From 0 index to 5th index, by 1, is"}\NormalTok{, myname[}\DecValTok{0}\NormalTok{:}\DecValTok{5}\NormalTok{:}\DecValTok{1}\NormalTok{])}
\end{Highlighting}
\end{Shaded}

\begin{verbatim}
## From 0 index to 5th index, by 1, is Davut
\end{verbatim}

\textbf{Short version}

\begin{Shaded}
\begin{Highlighting}[]
\BuiltInTok{print}\NormalTok{(}\StringTok{"Object myname:"}\NormalTok{, myname)}
\end{Highlighting}
\end{Shaded}

\begin{verbatim}
## Object myname: Davut Emrah Ayan
\end{verbatim}

\begin{Shaded}
\begin{Highlighting}[]
\BuiltInTok{print}\NormalTok{(myname[}\DecValTok{0}\NormalTok{:}\DecValTok{5}\NormalTok{], }\StringTok{"is the first 5 element of the object"}\NormalTok{)}
\end{Highlighting}
\end{Shaded}

\begin{verbatim}
## Davut is the first 5 element of the object
\end{verbatim}

\textbf{Shorter version}

Numeric string is easier to see.

\begin{Shaded}
\begin{Highlighting}[]
\NormalTok{num }\OperatorTok{=} \StringTok{"0123456789"}
\BuiltInTok{print}\NormalTok{(num[::}\DecValTok{2}\NormalTok{], }\StringTok{"every 2 other element"}\NormalTok{)}
\end{Highlighting}
\end{Shaded}

\begin{verbatim}
## 02468 every 2 other element
\end{verbatim}

\begin{Shaded}
\begin{Highlighting}[]
\BuiltInTok{print}\NormalTok{(num[::}\DecValTok{3}\NormalTok{], }\StringTok{"every 3 other element"}\NormalTok{)}
\end{Highlighting}
\end{Shaded}

\begin{verbatim}
## 0369 every 3 other element
\end{verbatim}

\begin{Shaded}
\begin{Highlighting}[]
\BuiltInTok{print}\NormalTok{(num[::}\DecValTok{4}\NormalTok{], }\StringTok{"every 4 other element"}\NormalTok{)}
\end{Highlighting}
\end{Shaded}

\begin{verbatim}
## 048 every 4 other element
\end{verbatim}

\begin{Shaded}
\begin{Highlighting}[]
\BuiltInTok{print}\NormalTok{(num[::}\DecValTok{5}\NormalTok{], }\StringTok{"every 5 other element"}\NormalTok{)}
\end{Highlighting}
\end{Shaded}

\begin{verbatim}
## 05 every 5 other element
\end{verbatim}

\textbf{Length} of an object : \texttt{len()}

it is character length or element length

\begin{Shaded}
\begin{Highlighting}[]
\BuiltInTok{len}\NormalTok{(myname)}
\end{Highlighting}
\end{Shaded}

\begin{verbatim}
## 16
\end{verbatim}

\hypertarget{concatenate}{%
\subsection{Concatenate}\label{concatenate}}

\begin{Shaded}
\begin{Highlighting}[]
\NormalTok{statement }\OperatorTok{=} \StringTok{"KU"} \OperatorTok{+} \StringTok{" is the best!"}  
\NormalTok{statement}
\end{Highlighting}
\end{Shaded}

\begin{verbatim}
## 'KU is the best!'
\end{verbatim}

\begin{Shaded}
\begin{Highlighting}[]
\NormalTok{myname[}\DecValTok{0}\NormalTok{:}\DecValTok{5}\NormalTok{] }\OperatorTok{+} \StringTok{" is the best!"}
\end{Highlighting}
\end{Shaded}

\begin{verbatim}
## 'Davut is the best!'
\end{verbatim}

\textbf{Multiplication with strings}

\begin{Shaded}
\begin{Highlighting}[]
\NormalTok{myname[}\DecValTok{0}\NormalTok{:}\DecValTok{6}\NormalTok{] }\OperatorTok{*} \DecValTok{3}
\end{Highlighting}
\end{Shaded}

\begin{verbatim}
## 'Davut Davut Davut '
\end{verbatim}

\textbf{Strings are Immutable}

\hypertarget{pandas-library}{%
\chapter{Pandas Library}\label{pandas-library}}

run \texttt{pip3\ install\ pandas}
on rstudio terminal or mac terminal or jupyter notebook

\hypertarget{read-data}{%
\section{read data}\label{read-data}}

\begin{Shaded}
\begin{Highlighting}[]
\ImportTok{import}\NormalTok{ pandas }\ImportTok{as}\NormalTok{ pd }
\NormalTok{pd.set\_option(}\StringTok{\textquotesingle{}display.max\_columns\textquotesingle{}}\NormalTok{, }\VariableTok{None}\NormalTok{)}
\end{Highlighting}
\end{Shaded}

\hypertarget{csv-file}{%
\subsection{csv file}\label{csv-file}}

\textbf{IBM sample data:} I could not run with ``https'' because I did not have a certificate installed. So, I go on with ``http'' and it worked.

\begin{Shaded}
\begin{Highlighting}[]
\NormalTok{data\_link }\OperatorTok{=} \StringTok{"http://s3{-}api.us{-}geo.objectstorage.softlayer.net/cf{-}courses{-}data/CognitiveClass/DS0103EN/labs/data/recipes.csv"}

\NormalTok{recipes }\OperatorTok{=}\NormalTok{ pd.read\_csv(data\_link)}
\end{Highlighting}
\end{Shaded}

\hypertarget{explore-data}{%
\section{Explore Data}\label{explore-data}}

\begin{Shaded}
\begin{Highlighting}[]
\CommentTok{\#recipes.head()}
\end{Highlighting}
\end{Shaded}

\hypertarget{get-the-dimensions-of-the-dataframe.}{%
\subsection{Get the dimensions of the dataframe.}\label{get-the-dimensions-of-the-dataframe.}}

\begin{Shaded}
\begin{Highlighting}[]
\CommentTok{\#recipes.shape}
\end{Highlighting}
\end{Shaded}

\hypertarget{random}{%
\chapter{RANDOM}\label{random}}

\begin{Shaded}
\begin{Highlighting}[]
\NormalTok{i }\OperatorTok{=} \BuiltInTok{int}\NormalTok{(}\BuiltInTok{input}\NormalTok{(}\StringTok{"sayi gir = "}\NormalTok{))}

\ControlFlowTok{for}\NormalTok{ x }\KeywordTok{in} \BuiltInTok{range}\NormalTok{(i}\OperatorTok{+}\DecValTok{1}\NormalTok{) :}
\NormalTok{  y }\OperatorTok{=}\NormalTok{ x}\OperatorTok{*}\DecValTok{5}
  \BuiltInTok{print}\NormalTok{(x,}\StringTok{"x"}\NormalTok{,}\DecValTok{5}\NormalTok{,}\StringTok{"="}\NormalTok{,y)}
\end{Highlighting}
\end{Shaded}

\hypertarget{applications}{%
\chapter{Applications}\label{applications}}

Some \emph{significant} applications are demonstrated in this chapter.

\hypertarget{example-one}{%
\section{Example one}\label{example-one}}

\hypertarget{example-two}{%
\section{Example two}\label{example-two}}

\hypertarget{final-words}{%
\chapter{Final Words}\label{final-words}}

We have finished a nice book.

  \bibliography{book.bib,packages.bib}

\end{document}
